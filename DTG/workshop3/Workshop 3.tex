\RequirePackage[l2tabu,orthodox]{nag}
\documentclass[a4paper,12pt]{article} 	% Openright aabner kapitler paa hoejresider (openany begge)

%%%% PACKAGES %%%%

\usepackage{color}

% ¤¤ Oversaettelse og tegnsaetning ¤¤ %
\usepackage[utf8]{inputenc}					% Input-indkodning af tegnsaet (UTF8)
\usepackage[british]{babel}					% Dokumentets sprog
\usepackage[T1]{fontenc}					% Output-indkodning af tegnsaet (T1)
\usepackage{hyperref}

% ¤¤ Matematik mm. ¤
\usepackage{amsmath,amssymb} 		% Avancerede matematik-udvidelser
\usepackage{mathtools}						% Andre matematik- og tegnudvidelser
\usepackage{setspace}
\usepackage[footnote,draft,danish,silent,nomargin]{fixme}
\usepackage[noend]{algpseudocode}

% ¤¤ Litteraturlisten ¤¤ %
\usepackage[backend=bibtex]{biblatex}


%Tikz%
\usepackage{tikz}
\usetikzlibrary{patterns}

\newcommand{\floor}[1]{\left\lfloor{#1}\right\rfloor}
\newcommand{\ceil}[1]{\left\lceil{#1}\right\rceil}

\begin{document}
\section*{Workshop 3 - Induktion}
I denne workshop kigger vi på \textsc{Mergesort}, som er en sorteringsalgoritme. Algoritmen i pseudokode kan ses i Figur \ref{fig:Mergesort} eller i afsnit 5.4.4 i [Ros]. \textsc{Mergesort} benytter \textsc{Merge}, som kan ses i Figur \ref{fig:Merge} eller i afsnit 5.4.4 i [Ros]. Bemærk at den her er beskrevet på en lidt anden måde, men at metoden og tanken bag er den samme. Inden I begynder på selve opgaverne kan I med fordel læse of forstå afsnit 5.4.4. Som algoritmen er beskrevet herunder vil man starte med at kalde \textsc{Mergesort}($L=(a_0,a_1,\ldots,a_{n-1}) $, $0$, $n-1$).

\begin{figure}[h!]
		\begin{algorithmic}
		\Procedure{Mergesort}{$L=(a_0,a_1,\ldots,a_{n-1}) $, $l$, $r$}
		\If{$l<r$} 
		\State $m=\floor{\frac{r+l}{2}}$
		\State $\textsc{Mergesort}(L, l, m)$
		\State $\textsc{Mergesort}(L, m+1, r)$
		\State $L= \textsc{Merge}(L, l, m,r)$\EndIf
		\Return $L$
		\EndProcedure
	\end{algorithmic}
	\caption{Mergesort}\label{fig:Mergesort}
\end{figure}

\begin{figure}[h!]
		\begin{algorithmic}
		\Procedure{Merge}{$L,l,m,r$}
		\State $L_1=L[l,l+1,\ldots,m]$
		\State $L_2=L[m+1,m+2,\ldots,r]$
		\State $i=0$
		\State $j=0$
			\While{$i<m-l+1$ and $j<r-m$}
				\If{$L_1[i]\leq L_2[j]$} \State $L[l+i+j]=L_1[i]$\State $i=i+1$
				\Else ~$L[l+i+j]=L_2[j]$\State $j=j+1$
				\EndIf
			\EndWhile
			\If{$i=m-l+1$} \For{$k=j\ldots r-m-1$}
							\State$L[l+i+k]=L_2[k]$
						\EndFor
			\Else \For{$k=i\ldots m-l$}
							\State$L[l+j+k]=L_1[k]$
						\EndFor
			\EndIf
			\Return $L$
		\EndProcedure
	\end{algorithmic}
	\caption{Merge}\label{fig:Merge}
\end{figure}

\subsubsection*{Delopgave 1}
\begin{enumerate}
	\item Implementer \textsc{Merge} og \textsc{Mergesort}. 
	\item Brug \textsc{Mergesort} til at sortere listen $L=(5,3,8,1,6,10,7,2,4,9)$
\end{enumerate}

\subsubsection*{Delopgave 2}
\begin{enumerate}
	\item Bevis ved hjælp af induktion, at \textsc{Mergesort} returnerer den sorterede liste hvis den får en liste med heltal som input. I beviset antages det, at \textsc{Merge} returnerer en sammenflettet sorteret liste, hvis dellisterne $L[l,l+1,\ldots,m]$ og $L[m+1,m+2,\ldots,r]$ er sorteret.
\end{enumerate}
Lemma 1 i afsnit 5.4.4 i [Ros] siger, at to sorterede lister med $m$ og $n$ elementer kan sammenflettes (merged) til en sorteret liste ved højst $m+n-1$ sammenligninger. 

\subsubsection*{Delopgave 3}
\begin{enumerate}
	\item Antag at \textsc{Merge} benytter $m+n-1$ sammenligninger til at sammenflette to lister med $m$ og $n$ elementer. Antag yderligere, at $n=2^k$ og vis ved induktion over $k$ (start med $k=0$), at \textsc{Mergesort} bruger præcis
	\begin{align*}
		n(\log_2(n)+1)=2^k(k+1)
	\end{align*}
	sammenligninger hvis input-listen $L$ har $n=2^k$ elementer. Hvilket slags induktionsbevis brugte du?
	\item Lav nu et induktionsbevis for, at \textsc{Mergesort} bruger mindre end eller lig med $2n\log_2(n)$ sammenligninger for alle $n\geq 2$ under samme antagelser om \textsc{Merge} som ovenfor. 
	
	Hints til opgaven: Vi bliver nødt til at lave induktion over $n$. Bemærk at en liste med $n+1$ elementer deles op i to lister med $\ceil{\frac{n+1}{2}}$ og $\floor{\frac{n+1}{2}}$ elementer i Mergesort. Desuden kan I få brug for følgende vurderinger:
	\begin{itemize}
		\item $2\floor{\frac{n+1}{2}}\log_2(\floor{\frac{n+1}{2}})\leq 2\floor{\frac{n+1}{2}}\log_2(\ceil{\frac{n+1}{2}})$
		\item $\floor{\frac{n+1}{2}}+\ceil{\frac{n+1}{2}}=\frac{n+1}{2}$
		\item $\ceil{\frac{n+1}{2}}\leq \frac{2}{3}(n+1)$ når $n$ er et positivt heltal
		\item $\log_2(\frac{2}{3}(n+1))=\log_2(n+1)-\log_2(\frac{3}{2})$
		\item $n+1-2(n+1)\log_2(\frac{3}{2})<0$ når $n$ er positiv.
	\end{itemize}	 
	
	\item Hvilket slags induktionsbevis brugte du? Og hvad siger resultatet om tidskompleksiteten/store-$O$ for mergesort?
\end{enumerate}

\end{document}